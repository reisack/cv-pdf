%% start of file `template.tex'.
%% Copyright 2006-2013 Xavier Danaux (xdanaux@gmail.com).
%
% This work may be distributed and/or modified under the
% conditions of the LaTeX Project Public License version 1.3c,
% available at http://www.latex-project.org/lppl/.


\documentclass[12pt,a4paper,sans]{moderncv}        % possible options include font size ('10pt', '11pt' and '12pt'), paper size ('a4paper', 'letterpaper', 'a5paper', 'legalpaper', 'executivepaper' and 'landscape') and font family ('sans' and 'roman')

% moderncv themes
\moderncvstyle{classic}                             % style options are 'casual' (default), 'classic', 'oldstyle' and 'banking'
\moderncvcolor{blue}                               % color options 'blue' (default), 'orange', 'green', 'red', 'purple', 'grey' and 'black'
%\renewcommand{\familydefault}{\sfdefault}         % to set the default font; use '\sfdefault' for the default sans serif font, '\rmdefault' for the default roman one, or any tex font name
\nopagenumbers{}                                  % uncomment to suppress automatic page numbering for CVs longer than one page

\usepackage[none]{hyphenat}

% character encoding
\usepackage[default]{sourcesanspro}
\usepackage[T1]{fontenc}
  
\usepackage[utf8]{inputenc}                       % if you are not using xelatex ou lualatex, replace by the encoding you are using
%\usepackage{CJKutf8}                              % if you need to use CJK to typeset your resume in Chinese, Japanese or Korean

% adjust the page margins
\usepackage[scale=0.90]{geometry}
%\setlength{\hintscolumnwidth}{3cm}                % if you want to change the width of the column with the dates
%\setlength{\makecvtitlenamewidth}{10cm}           % for the 'classic' style, if you want to force the width allocated to your name and avoid line breaks. be careful though, the length is normally calculated to avoid any overlap with your personal info; use this at your own typographical risks...
% personal data
\name{Rémi}{Eisack}
\title{Lead Developer .NET/TypeScript (Vue.js/Angular)}                               % optional, remove / comment the line if not wanted
\address{__ADDRESS_1__}{__ADDRESS_2__}{}% optional, remove / comment the line if not wanted; the "postcode city" and and "country" arguments can be omitted or provided empty
\phone[mobile]{__MOBILE__}                   % optional, remove / comment the line if not wanted
%\phone[fixed]{+2~(345)~678~901}                    % optional, remove / comment the line if not wanted
%\phone[fax]{+3~(456)~789~012}                      % optional, remove / comment the line if not wanted
\email{__EMAIL__}                               % optional, remove / comment the line if not wanted
\homepage{github.com/reisack}                       % optional, remove / comment the line if not wanted
\extrainfo{Permis B, véhicule}                 % optional, remove / comment the line if not wanted
%\photo[64pt][0.4pt]{picture}                       % optional, remove / comment the line if not wanted; '64pt' is the height the picture must be resized to, 0.4pt is the thickness of the frame around it (put it to 0pt for no frame) and 'picture' is the name of the picture file
%\quote{Some quote}                                 % optional, remove / comment the line if not wanted

% to show numerical labels in the bibliography (default is to show no labels); only useful if you make citations in your resume
%\makeatletter
%\renewcommand*{\bibliographyitemlabel}{\@biblabel{\arabic{enumiv}}}
%\makeatother
%\renewcommand*{\bibliographyitemlabel}{[\arabic{enumiv}]}% CONSIDER REPLACING THE ABOVE BY THIS

% bibliography with mutiple entries
%\usepackage{multibib}
%\newcites{book,misc}{{Books},{Others}}
%----------------------------------------------------------------------------------
%            content
%----------------------------------------------------------------------------------
\begin{document}
%\begin{CJK*}{UTF8}{gbsn}                          % to typeset your resume in Chinese using CJK
%-----       resume       ---------------------------------------------------------
\makecvtitle

\section{Compétences informatiques}
\cvitem{Langages}{C\# (.NET v4.7 à v8), TypeScript (v4/v5), JavaScript, T-SQL, HTML5/CSS3}
\cvitem{.NET}{ASP.NET Core (API), Entity Framework Core, ASP.NET WebForms}
\cvitem{TypeScript}{Vue.js (v3), PrimeVue, Angular (v9 à v17), Ionic (v5 à v8, Capacitor), TypeORM}
\cvitem{Tests unitaires}{NUnit, NSubstitute, Vitest, Jasmine}
\cvitem{Outils/plateformes}{Git, GitLab, Swagger, SonarQube, Sentry, Grafana Loki, Microsoft Azure, Teams}
\cvitem{Intégration}{GitLab CI/CD, Docker (Dockerfile, Docker Compose)}
\cvitem{Données/BI}{SQL Server (2016/2022), Integration Services (SSIS), Reporting Services (SSRS)}
\cvitem{Assistants IA}{ChatGPT Plus (Génération de code/documentation), Windsurf (auto-complétion)}

\section{Expériences professionnelles}
\cventry{Depuis 2022}{Lead Developer}{SMAG}{Montpellier}{}{Responsable technique d'une équipe de 4 développeurs, sur la partie végétale de l'ERP vendu à des coopératives agricoles.
\begin{itemize}%
\item Développements et modernisation
  \begin{itemize}%
  \item Applications frontend Vue.js 3 + TypeScript et API .NET 8 (back-for-front) pour remplacer progressivement des écrans ASP.NET WebForms et ASP Classic de l'ERP
  \item Interaction avec SQL Server 2016 et 2022 via EF Core
  \item Réalisation de migrations régulières de frameworks et de packages
  \end{itemize}
\item Maintenance applicative sur l'ERP en ASP.NET WebForms 4.7 et T-SQL
\item Qualité logicielle
  \begin{itemize}%
  \item Supervision de la qualité des tests unitaires dans l'équipe
  \item Mise en place de SonarQube sur les projets de l'équipe, avec un minimum de 80 \% de couverture sur le nouveau code, datant de moins de 90 jours
  \item Mise en place de Sentry pour surveiller les avertissements et erreurs applicatives
  \end{itemize}
\item Leadership technique
  \begin{itemize}%
  \item Animation de réunions hebdomadaires d'amélioration technique au sein de l'équipe
  \item Rédaction de la documentation technique des projets en Markdown/Mermaid, dans les wikis des repositories GitLab
  \item Revues de code systématiques et pair programming selon les besoins
  \end{itemize}
\item Collaboration transverse
  \begin{itemize}%
  \item Communication régulière avec l'équipe DevOps, notamment via un point hebdomadaire (référent DevOps de l’équipe)
  \item Participation à un point hebdomadaire sur le partage et l'amélioration de l'utilisation des technologies backend dans l'entreprise
  \end{itemize}
\item Développement des aspects CI des projets de l'équipe (GitLab CI), pour automatiser l'exécution des tests unitaires et la création des livrables
\item Fonctionnement d'équipe en mode agile selon la méthodologie SCRUM et utilisation de JIRA pour la gestion des sprints
\item Participation ponctuelle au processus de recrutement via la conduite d'entretiens techniques (environ 10 candidats pour 4 recrutements finalisés avec succès)
\end{itemize}}
\cventry{2018--2022}{Ingénieur développement}{SMAG}{Montpellier}{}{Développements concernant la partie végétale de l'ERP vendu à des coopératives agricoles.
\begin{itemize}%
\item Développements en ASP.NET 4.7 WebForms et T-SQL
\item Maintenance et ajout de fonctionnalités sur une application mobile basée sur Ionic (Angular/TypeScript, Cordova), concernant la gestion de parcelles agricoles
\item Mise en place de tests unitaires pour renforcer la qualité du code et limiter les anomalies
  \begin{itemize}%
  \item Utilisation du framework Jasmine pour l'application mobile en Ionic/Angular
  \item Expérimentation de tests unitaires sur du code legacy en .NET avec MSTest et Microsoft Fakes, en l'absence d'une architecture adaptée aux tests unitaires standards
  \end{itemize}
\item Expérience en télétravail à 100 \% pendant un an (crise sanitaire), puis 3 jours par semaine
\item Promotion au poste de Lead Developer grâce aux résultats obtenus en matière de qualité logicielle et à l'adoption des bonnes pratiques dans l'équipe
\end{itemize}}
\cventry{2013--2018}{Développeur .NET}{Acelys}{Montpellier}{}{
\begin{itemize}%
\item Développement de diverses applications en WPF (.NET 4.0)
\item Développement d'une application de gestion d'assurances pour le personnel médical en ASP.NET Web API (back-end) et AngularJS (front-end). Utilisation de l'ORM NHibernate.
\item A partir de novembre 2015 : mission chez SMAG, éditeur de logiciels fournissant des solutions destinées aux coopératives agricoles
  \begin{itemize}%
  \item Développements en ASP.NET 4.0 WebForms et T-SQL (Procédures stockées)
  \item Création et mise à jour de divers rapports avec Reporting Services 2008 R2, pour répondre à des besoins spécifiques de coopératives agricoles
  \item Développement d'une application web mobile à l'aide du framework Sencha Touch 2.4, pour un grand groupe souhaitant assurer la transparence et la pérennité de son approvisionnement en caoutchouc naturel
  \item Utilisation d'Integration Services 2008 R2 (de mai 2016 à octobre 2016) dans un cadre ETL pour deux coopératives agricoles, clientes de SMAG. Afin de fusionner leurs bases de données respectives en une seule
  \end{itemize}
\end{itemize}}
\cventry{2010--2013}{Développeur .NET}{Aessentia}{Lyon}{}{Développements sur l'intranet du groupe La Lyonnaise des Eaux (aujourd'hui Suez Eau France).
\begin{itemize}%
  \item Développements en C\# (.NET 4.0, WebForms) avec les composants UI Telerik ASP.NET Ajax
  \item Création et mise à jour de rapports avec Reporting Services 2008 R2, sur les données de qualité de l'eau et les travaux d'assainissement
  \item Mise en place d'un mécanisme de génération de fichiers Word et Excel 2007 sur l'intranet via Open XML SDK, permettant aux collaborateurs de La Lyonnaise des Eaux de générer des modèles de documents administratifs basés sur les données de l'intranet
  \item Participation au déploiement de l'application et à la création du package d'installation
\end{itemize}}
\cventry{2009--2010}{Développeur PHP}{Avanim Production}{Lyon}{}{Développement de sites web de rencontre et d'e-commerce en PHP.
\begin{itemize}%
  \item Développements en PHP 4 et 5
  \item Création et intégration de modules et de templates sur Joomla 1.0 et 1.5, en personnalisant le framework selon les besoins
  \item Personnalisation des composants JomSocial (sites de rencontre) et VirtueMart (e-commerce)
  \item Utilisation de jQuery pour ajouter des interactions et animations dynamiques aux sites web
\end{itemize}}

\section{Formations/Certifications Microsoft}

\cventry{2017}{Examen 70-483 : Programming in C\#}{}{}{}{}
\cventry{}{Examen 70-480 : Programming in HTML5 with JavaScript and CSS3}{}{}{}{}

\cventry{2008--2009}{Licence Professionnelle informatique embarquée et mobile}{Université Lyon 1 Claude Bernard}{Bourg En Bresse}{}{}  % arguments 3 to 6 can be left empty
\cventry{2006--2008}{BTS Informatique de gestion option Développeur d'applications}{Lycée Lamartinière - Duchère}{Lyon}{}{}
% Publications from a BibTeX file without multibib
%  for numerical labels: \renewcommand{\bibliographyitemlabel}{\@biblabel{\arabic{enumiv}}}% CONSIDER MERGING WITH PREAMBLE PART
%  to redefine the heading string ("Publications"): \renewcommand{\refname}{Articles}
%\nocite{*}
%\bibliographystyle{plain}
%\bibliography{publications}                        % 'publications' is the name of a BibTeX file

% Publications from a BibTeX file using the multibib package
%\section{Publications}
%\nocitebook{book1,book2}
%\bibliographystylebook{plain}
%\bibliographybook{publications}                   % 'publications' is the name of a BibTeX file
%\nocitemisc{misc1,misc2,misc3}
%\bibliographystylemisc{plain}
%\bibliographymisc{publications}                   % 'publications' is the name of a BibTeX file

\clearpage


\end{document}


%% end of file `template.tex'.
